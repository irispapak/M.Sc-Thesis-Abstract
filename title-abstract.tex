\documentclass[greek,12pt]{report}
\usepackage[utf8]{inputenc}
\usepackage[english]{babel}
\newcommand{\en}{\selectlanguage{english}}
\newcommand{\gr}{\selectlanguage{greek}}

\usepackage{graphicx}
\graphicspath{ {/home/irida/Documents/Thesis/images/} }

\usepackage[toctitles]{titlesec}
\usepackage{titlesec, blindtext, color}
\definecolor{gray75}{gray}{0.75}
\newcommand{\hsp}{\hspace{20pt}}
\titleformat{\chapter}[hang]{\Huge\bfseries}{\thechapter\hsp\textcolor{gray75}{$|$}\hsp}{0pt}{\Huge\bfseries}

\setlength{\parindent}{2em}
\setlength{\parskip}{0.8em}
\renewcommand{\baselinestretch}{1.2}
\usepackage{indentfirst}
\usepackage[a4paper, width=150mm, top=25mm, bottom=25mm, bindingoffset=6mm]{geometry}

% bibliography
\usepackage{natbib}
\bibliographystyle{plainnat}
\addto{\captionsenglish}{\renewcommand{\bibname}{\gr \textsc{Βιβλιογραφία}}}
\setcitestyle{authoryear, open={(},close={)}}

\newcommand{\mychapter}[1]{\textsc{\chapter{#1}}}

\usepackage{makecell}
\renewcommand{\arraystretch}{1.0}
\usepackage[font={footnotesize}]{caption}

% header/footer
\usepackage{fancyhdr}
\pagestyle{fancy}
\fancyhf{}
\fancyhead[RE,RO]{\it \leftmark}
\fancyhead[LE,LO]{\it \rightmark}
\fancyfoot[CE,CO]{\thepage}

\usepackage{datatool}
\usepackage{csvsimple}

\usepackage{hyperref}
\hypersetup{
    colorlinks=true,
    linkcolor=blue,
    filecolor=blue,      
    urlcolor=cyan,
    citecolor=PineGreen,
  }

\usepackage[dvipsnames,table]{xcolor}
\usepackage{gensymb}
\usepackage[LGR, T1]{fontenc}


\begin{document}


\begin{titlepage}
  \begin{center}
  
    \vspace*{.01\textheight}
    {\scshape\LARGE Aristotle University of Thessaloniki \par}%\vspace{1.5cm} % University name
    \includegraphics[width=4cm,keepaspectratio]{auth.png} % University/department logo - uncomment to place it
    \par
    \textsc{\Large School of Geology\\ \vspace{0.3cm}Department of Meteorology and Climatology} 
    \vspace{1cm}
    \hrule \vspace{0.5cm}
    % Horizontal line
    {\Large \bfseries Investigating the added-value of high-resolution regional climate simulations over Europe \par}\vspace{0.7cm} % Thesis title
    \hrule \vspace{1.5cm} % Horizontal line
    
    \textsc{\large M.Sc Thesis}\\[1.5cm] % Thesis type
    {\large \bfseries Iris Papakonstantinou-Presvelou} 
    \vspace{4cm}
    
    {\large Thessaloniki, 2019}%\\[2cm] % Date
    
  \end{center}
\end{titlepage}

\newpage

\chapter*{\textsl{Abstract}} \thispagestyle{empty}

In the current study the effect of spatial resolution in regional climate simulations is being investigated. The focus is on the added value of high-resolution climate simulations, which are performed in the framework of EURO-CORDEX. The analysis includes EURO-CORDEX simulations in three different domain resolutions: coarse ($\sim$50km), fine ($\sim$12km) and convective-permitting (hereafter CP) ($\sim$3km). The coarse and fine resolution simulations cover the whole European domain, while the CP is over the greater Alpine region. An evaluation of the mean, maximum and minimum temperature and precipitation is performed, calculating basic performance metrics (e.g. bias) and identifying the regions which deviate from observed values. The evaluation datasets used in this study are: E-OBs v.19 for temperature and precipitation (12km), the MESAN reanalysis for temperature (5km) and the EURO4M-APGD for precipitation (5km). The results show that during the warm season in the finer resolution simulations at 12km compared to the coarser, the bias in maximum and mean temperature improves by 0.7\degree C and 0.3\degree C respectively, but deteriorates for minimum temperature by 0.3\degree C (over all domain). During the cold season, the biases differ only for mean temperature (deteriorate by 0.6\degree C in finer resolution), but for maximum and minimum temperature the absolute bias remains almost unchanged. Concerning precipitation over the Alps, the biases don't change much at the CP domain (3km) when compared to the finer domain during winter, spring and autumn, but in summer the model becomes considerably drier (bias increases by 24\%).

\end{document}


